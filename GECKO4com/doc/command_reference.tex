%%%%%%%%%%%%%%%%%%%%%%%%%%%%%%%%%%%%%%%%%%%%%%%%%%%%%%%%%%%%%%%%%%%%%%%%%%%%%%%%
%%            _   _            __   ____                                      %%
%%           / / | |          / _| |  __|                                     %%
%%           | |_| |  _   _  / /   | |_                                       %%
%%           |  _  | | | | | | |   |  _|                                      %%
%%           | | | | | |_| | \ \_  | |__                                      %%
%%           |_| |_| \_____|  \__| |____| microLab                            %%
%%                                                                            %%
%%           Bern University of Applied Sciences (BFH)                        %%
%%           Quellgasse 21                                                    %%
%%           Room HG 4.33                                                     %%
%%           2501 Biel/Bienne                                                 %%
%%           Switzerland                                                      %%
%%                                                                            %%
%%           http://www.microlab.ch                                           %%
%%%%%%%%%%%%%%%%%%%%%%%%%%%%%%%%%%%%%%%%%%%%%%%%%%%%%%%%%%%%%%%%%%%%%%%%%%%%%%%%
\chapter{Implemented Commands}
\label{chap:commands}
The {\sc GECKO4com} implements 36 {\sc SCPI} commands as shown in
Table~\ref{tab:scpi commands}. The commands starting with a \verb+*+ are
compliant with the IEEE488.1~\cite{ieee488_1} and IEEE488.2~\cite{ieee488_2}
standards and will not be described in this chapter. The exception to the above
is the description of the \verb+*PUD+, \verb+*PUD?+ and \verb+*RST+ commands.
\begin{table}[hb]
\begin{tabular}{|l|l|l|l|}
\hline
\verb+*CLS+&\verb+*ESE+&\verb+*ESE?+&\verb+*ESR?+\\
\hline
\verb+*IDN?+&\verb+*IST?+&\verb+*OPC+&\verb+*OPC?+\\
\hline
\verb+*PUD+&\verb+*PUD?+&\verb+*RST+&\verb+*SRE+\\
\hline
\verb+*SRE?+&\verb+*STB?+&\verb+*TST?+&\verb+*WAI+\\
\hline
\verb+BITFLASH+&\verb+BITFLASH?+&\verb+BOARD?+&\verb+CONFIG+\\
\hline
\verb+ERASE+&\verb+FIFO+&\verb+FIFO?+&\verb+FPGA+\\
\hline
\verb+FPGA?+&\verb+HEXSWITCH+&\verb+HEXSWITCH?+&\verb+IDENTIFY+\\
\hline
\verb+TRANS+&\verb+USERRESET+&\verb+VGA:BGCOL+&\verb+VGA:CLEAR+\\
\hline
\verb+VGA:CURSOR+&\verb+VGA:CURSOR?+&\verb+VGA:FGCOL+&\verb+VGA:PUTSTR+\\
\hline
\end{tabular}
\caption{{\sc SCPI} commands implemented in the {\sc GECKO4com}}
\label{tab:scpi commands}
\end{table}
%-----------------------------------------------------------------------------
\section{Protected User Data}
\label{sec: PUD}
The IEEE488.1~\cite{ieee488_1} and IEEE488.2~\cite{ieee488_2} standards describe
the \verb+*PUD+ and \verb+*PUD?+ as optional commands. These two commands
provide a Protected User Data (PUD) area of at least 63 bytes. The {\sc GECKO4com}
implements a PUD of 2048 bytes and violates the standard in the protection
requirement.\\
 \\
\textbf{Reading the PUD area:}\\
When executing the \verb+*PUD?+ command the {\sc GECKO4com} puts the contents of
the complete PUD memory (2048 bytes) in the output queue. The PUD memory can
only be read completely; there exist no possibility to read only a part of it.\\
 \\
\textbf{Writing the PUD area:}\\
The PUD area can be written with the \verb*+*PUD <payload>+ command. The number
of spaces (\verb*+ +) between the command and the payload must be exactly
one. All the other spaces are interpreted as payload. The size of
the payload can be anywhere between 1 and 2048 bytes. Upon receiving this
command the {\sc GECKO4com} will start writing at address 0 of the PUD memory.
If the payload size is more than 2048 bytes the {\sc GECKO4com} will wrap around
and overwrite earlier written data. After having received the payload, the {\sc
GECKO4com} will store the complete 2048 bytes of the PUD memory in an external
attached flash.\\
\textit{Note:  If there is no Flash chip mounted on the {\sc GECKO4main}, the
PUD memory area will become volatile.\note}\\
\textit{Important: Each execution of the} \verb+*PUD+ \textit{command will initiate a
sector erase/program cycle of the attached Flash chip; therefore, the number of 
the} \verb+*PUD+ \textit{command invocation is limited by the number of 
sector erase cycles specified by the Flash chip manufacturer.\important}\\
 \\
\textbf{PUD memory organization:}\\
The PUD memory organization is shown in Figure~\ref{fig:PUD mem org}.
\begin{figure}
\centering%
\includegraphics[width=0.7\columnwidth]{figs/pud_memory_org}
\caption{Memory organization of the PUD area. The memory is continues; the
64-byte blocks are only shown to indicate the thirteen ASCII lines.}
\label{fig:PUD mem org}
\end{figure}
The area between \verb+0x000+ and \verb+0x4BF+ is completely user definable;
however the area between \verb+0x4C0+ and \verb+0x7FF+ is used to store thirteen
lines of sixty four ASCII characters that are displayed in the Static User Window
 (SUW) of the VGA-controller (see Chapter~\ref{sec: vga}).\\
  \\
\textbf{PUD command error handling:}\\
Both the \verb+*PUD+ and \verb+*PUD?+ are always executed successfully; therefore
there is no command error generated for either of these commands.
%-----------------------------------------------------------------------------
\section{Global Reset}
The {\sc GECKO4com} provides the means to activate the Global Reset. It is
important to not confuse the Global Reset with the User Reset as described in
Chapter~\ref{chap:user_reset}. The {\sc GECKO4com} activates the Global Reset by
pulling the signal actively low. In case of a deactivated Global Reset, the
{\sc GECKO4com} leaves this pin floating. No pull-up on this signal is provided
by the {\sc GECKO4com}. For more details on the requirements of the Global Reset
(GR) signal please refer to the {\sc GECKO4main} technical reference
guide~\cite{gecko4main}.\\
 \\
\textbf{GR activation:}\\
There are two situations for which the {\sc GECKO4com} activates the Global
Reset line, namely:
\begin{enumerate}
\item \textbf{Unconfigured FPGA:} If the user FPGA on the {\sc GECKO4main} is
not configured the Global Reset line is kept activate.
In case no user FPGA is mounted, the the Global Reset line is deactivated.
\item \textbf{Reset command:} By executing the \verb+*RST+ command, the Global
Reset line is activated for a period between 10~ms and 11~ms.
\end{enumerate}
\textit{Important: The {\sc GECKO4com} is uneffected by the state of the Global Reset
line.\important}\\
 \\
\textbf{GR command error handling:}\\
The \verb+*RST+ command always executes correctly; therefore there is no command
error generated for this command.
%-----------------------------------------------------------------------------
\section{User Reset}
\label{chap:user_reset}
Contrary to the Global Reset signal the User Reset signal is only connected to
the user FPGA on the {\sc GECKO4main}. This User Reset signal provides the means
of resetting the logic inside the user FPGA without effecting attached
boards. The User Reset (UR) signal is an \emph{active low} signal that is actively
driven for both a \verb+1+ and a \verb+0+. The User Reset signal is connected to
pin \textbf{Y9} of the user FPGA.\\
 \\
\textbf{UR activation:}\\
There are two situations for which the User Reset line is activated, namely:
\begin{enumerate}
\item \textbf{Unconfigured FPGA:} If the user FPGA on the {\sc GECKO4main} is
not configured the User Reset line is kept activate up to a period of 10~ms to
11~ms after the done line of the user FPGA went high.
In case no user FPGA is mounted, the User Reset line is deactivated.
\item \textbf{Reset command:} By executing the \verb+USERRESET+ command, the
User Reset line is activated for a period between 10~ms and 11~ms.
\end{enumerate}
\textit{Important: The {\sc GECKO4com} is uneffected by the state of the User Reset
line.\important}\\
\textit{Important: The User Reset line is uneffected by the state of the Global
Reset line.\important}\\
 \\
\textbf{UR command error handling:}\\
The \verb+USERRESET+ command always executes correctly; therefore there is no command
error generated for this command.
%-----------------------------------------------------------------------------
\section{Bitfile Storage}
To be able to operate the {\sc GECKO4main} in an environment where a USB
connection is not wanted, the {\sc GECKO4com} provides the means to store the
user FPGAs configuration in a non-volatile memory. The user FPGAs configuration,
further referred to as \emph{bitfile}, is generated by the Xilinx toolchain and
consists of a variable length header and the configuration data in form of a
bitstream~\cite{FPGAFAQ_0026}. The {\sc GECKO4com} provides four commands to
manipulate the bitfile storage.
\begin{itemize}
\item \verb*+BITFLASH <bitfile>+. This command stores the \verb+<bitfile>+ in
the non-volatile memory of the {\sc GECKO4com}. This command can take upto two
seconds to complete due to the speed of the non-volatile memory. This command
is only succesfull in case of an empty non-volatile memory, otherwise an
execution error is generated.\\
\textit{Important: (1) The {\sc GECKO4com} does not control if the provided
bitfile is suitable for the mounted user FPGA and
(2) only a single space (}\verb*+ +\textit{)  must be provided between the command and
the bitfile.}
\item \verb+CONFIG+. This command configures the user FPGA with the bitfile
stored in the non-volatile memory. If the non-volatile memory does not contain a
bitfile an execution error is generated.
\item \verb+ERASE+. This command erases the non-volatile memory and must be
provided before the programming of a new bitfile into the non-volatile memory.
This command can take several seconds to complete.
\item \verb+BITFLASH?+. This command returns \verb+EMPTY+ if the non-volatile
memory is empty, and the stored bitfile otherwise.
\end{itemize}
\textit{Note: It is highly discouraged to use the }\verb+BITFLASH <bitfile>+
\textit{ and }
\verb+ERASE+ \textit{ commands in a script due to their execution time. Most likely a
time-out error is generated on the USBTMC protocol due to these execution
times. \note}
%-----------------------------------------------------------------------------
\section{{\sc GECKO4main} information}
\label{sec: main info}
The {\sc GECKO4com} provides two commands to report the current status of the
{\sc GECKO4main} system. Both these commands always execute successfully and
return an ASCII string.
\begin{itemize}
\item \verb+FPGA?+. This command returns the detected user FPGA type and its
configuration state. Possible user FPGAs are listed in the {\sc GECKO4main}
Technical Reference Manual~\cite{gecko4main}. If no user FPGA is
mounted or the {\sc GECKO4com} was unable to determine the FPGA type this
command returns the message \emph{No FPGA mounted or unknown FPGA type} and all
user FPGA manipulations are prohibited.
\item \verb+BOARD?+. This command returns the powering status, e.g. USB
supplied, GENIO1 supplied, powering BUS, of the {\sc GECKO4main} and the status
of the {\sc GECKO4com}'s non-volatile memory (empty or programmed) [see previous
section].
\end{itemize}
Furthermore the {\sc GECKO4com} provides the means of identifying optically an
attached {\sc GECKO4main} by means of the command \verb+IDENTIFY+. After
execution of this command the eight LEDs of {\sc GECKO4main} will start to
become red, afterwards they change to green and finally they will go off one
after the other. This sequence lasts for approximately one second.
%-----------------------------------------------------------------------------
\section{User FPGA configuration}
The {\sc GECKO4com} provides the means to configure the user FPGA. This
configuration is independent of the current state of the user FPGA. This means
that the user FPGA can be configured as many times as required and with as many
bitfiles as wanted. By this flexibility of the {\sc GECKO4com} progression
testing and scripted simulations/emulations. The configuration time for even the
largest Spartan 3 5000 FPGA is only a few tens of milliseconds.

The command to configure the user FPGA is \verb*+FPGA <bitfile>+. This command
\textbf{must} only contain one space (\verb*+ +). In case no user FPGA is
mounted or the {\sc GECKO4com} was unable to detect the mounted user FPGA  this
command will generate an execution error.\\
\textit{Important: The {\sc GECKO4com} does not control if the provided bitfile
is suitable for the mounted user FPGA. It is up to the user to make sure that
the bitfile that is uploaded is generated for the correct user FPGA type. In
case a wrong bitfile is uploaded the command will execute successfully; however
the user FPGA will refuse to load it. To be able to see if the user FPGA has
been configured correctly the }\verb+FPGA?+\textit{ command can be used to ask
the current status if the user FPGA. \important}
%-----------------------------------------------------------------------------
\section{Mechanical Switch}
\label{sec: hexswitch}
The {\sc GECKO4main} provides a hexadecimal encoded switch that is further
referred to as \emph{hexswitch}. The {\sc GECKO4com} provides two commands to
manipulate this switch.
\begin{itemize}
\item \verb*+HEXSWITCH <value>+. This command overrides the mechanical switch and
forces the \verb+<value>+ to appear on both the user FPGA side (see
Chapter~\ref{sec:mem map}) as well as from the PC side. The parameter
\verb+<value>+ can be any of the set [0-9,A-F,a-f]. Between the command and the
parameter \verb+<value>+ there may be as many spaces (\verb*+ +) as wanted. This
command executes successfully as long as the parameter \verb+<value>+ is in the
set mentioned before. This command is persistent until either the board is
powered, or an \verb+INITIATE_CLEAR+ USBTMC message~\cite{usbtmc} is send. This
command may be send as often as required and will each time override the
previous value send.
\item \verb+HEXSWITCH?+. This command reports the current state of the hexswitch
or the override value if a \verb+HEXSWITCH+ command has been issued before this
command. The result of this command is an ASCII character of the set [0-9,A-F]
followed by a new-line character (\verb+0x0A+).
\end{itemize}
%-----------------------------------------------------------------------------
\section{User FPGA communication}
The {\sc GECKO4com} provides three commands that allows for communication
between the PC and the user FPGA. These commands are \verb+FIFO+, \verb+FIFO?+,
and \verb+TRANS+. More information on these commands and the protocols involved
can be found in Chapter~\ref{chap:fpga}.
%-----------------------------------------------------------------------------
\section{VGA controller}
\label{sec: vga commands}
The {\sc GECKO4com} provides a VGA controller not only for debug purposes but
also for demonstration purposes. The VGA controller implemented in the {\sc
GECKO4com} has one USBTMC window (see Chapter~\ref{sec: vga} for further
information). The commands to manipulate this window are listed below.
\begin{itemize}
\item \verb*+VGA:FGCOL <value>+. This command sets the forground color of the
VGA window. The parameter \verb+<value>+ can be any character between \verb+0+
and \verb+7+ (see also Chapter~\ref{sec: vga}).
\item \verb*+VGA:BGCOL <value>+. This command sets the background color of the
VGA window. The parameter \verb+<value>+ can be any character between \verb+0+
and \verb+7+ (see also Chapter~\ref{sec: vga}).
\item \verb+VGA:CLEAR+. This command clears the VGA window and puts the cursor
on the left top position of the screen.
\item \verb+VGA:CURSOR?+. This command returns the current cursor position. The
format returned is a string in the form \verb+<x>,<y>+ where
$\text{x}\in[0..63]$ and $\text{y}\in[0..31]$.
\item \verb*+VGA:CURSOR <x>,<y>+. This command sets the current cursor position
to (x,y). The parameters: $\text{x}\in[0..63]$ and $\text{y}\in[0..31]$.\\
\textit{Note: The parameters x and y must be specified as ASCII characters and
not as bytes, e.g. 31 equals the character ``3'' (0x33) followed by the
character ``1'' (0x31).\note}
\item \verb*+VGA:PUTSTR <string>+. This command displays all the bytes following
the space (\verb*+ +) at the current cursor position on the VGA window as ASCII
characters. A new-line character (\verb+0x0A+) will forse a new line and all
bytes with a value smaller than the ASCII character space (\verb+0x20+) will be
ignored.\\
\textit{Important: No commands can be concatinated with this command. Doing so
will ``print'' them on the VGA screen rather than executing them.\important}
\end{itemize}
%-----------------------------------------------------------------------------
\section{The Status Byte Register}
\label{sec: SBR}
The IEEE488.1~\cite{ieee488_1} and IEEE488.2~\cite{ieee488_2} standards define
an obligatory status byte register as shown in Figure~11-8 in the
IEEE488.2~\cite{ieee488_2} standard. The {\sc GECKO4com} implements this
register with the obligatory \emph{MSS}, \emph{ESB}, and \emph{MAV} bits.
Besides from these obligatory bits, the {\sc GECKO4com} implements 2 extra
status bits.
\begin{itemize}
\item \textbf{Bit 3.} Bit 3 in the Status Byte Register indicates when \verb+1+
that the user FPGA is configured and running. When this bit is \verb+0+ the user
FPGA is either not present or not configured.
\item \textbf{Bit 2.} Bit 2 in the Status Byte Register indicates when \verb+1+
that the {\sc GECKO4com} is in \emph{transparent mode} (see Chapter~\ref{sec:trans mode}).
When this bit is \verb+0+ the {\sc GECKO4com} is in normal operation mode.
\end{itemize}
If the {\sc GECKO4com} is in \emph{transparent mode} it does only supply the
\emph{MSS} bit, the \emph{MAV} bit, and the transparent bit (bit 2) of the
Status Byte Register. It is left to the user FPGA to provide the following bits:
\begin{itemize}
\item \textbf{Bit 7 (ESB).} In \emph{transparent mode} the user FPGA {\sc must}
provide the \emph{ESB} bit. This bit is at pin \verb+AE6+ of the user FPGA and
of type {\sc lvcmos25}.
\item \textbf{Bit 3.} In \emph{transparent mode} this bit is user definable by
the user FPGA. This bit is at pin \verb+AE8+ of the user FPGA and of type {\sc
lvcmos25}.
\end{itemize}
For the user FPGA to know if the {\sc GECKO4com} is in \emph{transparent mode},
the pin \verb+AD6+ at the user FPGA is a copy of bit 2 in the Status Byte
Register. Figure~\ref{fig:stb} shows the Status Byte Register as provided by the
{\sc GECKO4com}.
\begin{figure}[t]
\centering%
\includegraphics[width=0.8\columnwidth]{figs/status_byte_register}
\caption{The Status Byte Register as defined by the {\sc GECKO4com}.}
\label{fig:stb}
\end{figure}
%-----------------------------------------------------------------------------
\def\srt#1{}
\begin{thebibliography}{10}
\bibitem{FPGAFAQ_0026}
Alan Nishioka and Philip Freidin,
\newblock{\em FPGA-FAQ 0026---{T}ell me about the .{BIT} file format.}
\newblock \verb+www.fpga-faq.com+, Nov. 2001

\bibitem{gecko4main}
Dr. Theo Kluter,
\newblock{\em {\sc GECKO4main} {T}echnical {R}eference {M}anual.}
\newblock Bern University of Applied Sciences, Biel/Bienne, Switzerland, 2011.

\bibitem{ieee488_1}
IEEE, Inc.
\newblock{\em {IEEE} {S}tandard {C}odes, {F}ormats, {P}rotocols, and
{C}ommon {C}ommands for {U}se {W}ith {IEEE} {S}td 488.1-1987, {IEEE}
{S}tandard {D}igital {I}nterface for {P}rogrammable {I}nstrumentation.}
\newblock ISBN 1-55937-238-9, New York, 1992.

\bibitem{ieee488_2}
IEC and IEEE.
\newblock{\em {IEC} 60488-2, {IEEE} 488.2: {S}tandard digital interface for
programmable instrumentation --- {P}art 2: {C}odes, formats, protocols and
common commands.}
\newblock First edition, 2005-05.

\bibitem{usbtmc}
USB Implementers Forum, Inc.
\newblock{\em {U}niversal {S}erial {B}us {T}est and {M}easurement {C}lass
{S}pecification ({USBTMC})--{R}evision 1.0.}
\newblock April, 2003
\end{thebibliography}
